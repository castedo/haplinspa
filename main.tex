\documentclass{article}
\usepackage{amsmath, amssymb}
\usepackage{hyperref}
\usepackage[backend=biber, style=numeric]{biblatex}
\addbibresource{references.bib}

\begin{document}


\newcommand{\mathstop}{\text{ .}}
\renewcommand{\Pr}{\operatorname{\mathbb{P}}}
\newcommand{\dom}{\operatorname{dom}}
\newcommand{\rng}{\operatorname{rng}}
\newcommand{\Tim}{\mathrm{Tim}}
\newcommand{\Hap}{\mathrm{Hap}}
\newcommand{\Dip}{\mathrm{Dip}}
\newcommand{\Liv}{\mathrm{Liv}}
\newcommand{\Par}{\mathrm{Par}}
\newcommand{\Mom}{\mathrm{Mom}}
\newcommand{\Fert}{\mathrm{Fert}}
\newcommand{\Loc}{\mathrm{Loc}}
\newcommand{\Lin}{\mathsf{Lin}}
\newcommand{\Leg}{\mathsf{Leg}}
\newcommand{\Nodes}{\mathsf{Nodes}}


\begin{abstract}
\textbf{STAGE:} REALLY ROUGH WORKING DRAFT

\textbf{DOCUMENT TYPE:} Formal Mathematical Definition

\textbf{OBJECTIVES:}

\begin{itemize}
\item
  Formal mathematical definition of stochastic process used by
  statistical estimator of admixture timing under development.
\item
  Precise mathematical definitions for technical discussions relating to
  ancestral recombination graphs.
\end{itemize}
\end{abstract}


\section{Sets of Indices}

A haploid lineage process is defined in terms of three sets of indices:
\begin{enumerate}
\item
  a set $\Tim$ of numerical time indices (e.g. integers or real numbers),
\item
  a set $\Dip$ of indices to diploid organisms, and
\item
  a set $\Hap$ of indices to haploid genomes (e.g. gametes).
\end{enumerate}

Although $\Tim$ can be chosen to be many possible sets of numbers, once $\Tim$
is given, we define:
$$
\Dip := \bigcup_{t \in \Tim} \Dip_t
$$
where for all $t \in \Tim$,
$$
\Dip_t := \{ (t, 0), (t, 1), (t, 2), (t, 3), ... \}
\mathstop
$$
Likewise we define
$$
\Hap := \bigcup_{t \in \Tim} \Hap_t
$$
where for all $t \in \Tim$,
$$
\Hap_t := \Dip_t \times \{ 0, 1 \}
$$
and $0$ and $1$ index egg and sperm, respectively.

For convenience, for every $t \in \Tim$ we define fertilization time as
$$
\Fert(x) := t
$$
for all $x \in \Dip_t \cup \Hap_t$.


\section{Haploid lineage process}

Given 
\begin{itemize}
\item
  a time index set $\Tim$ (with resultant $\Dip$, $\Hap$, $\Fert$),
\item
  a set $\Loc$ of genomic locations,
\item
  and probability space $(\Omega, \mathcal{F}, \Pr)$,
\end{itemize}
a \emph{Haploid lineage process} is a stochastic process defined by three time-indexed
random variables
$$
 \left\{ (\Par_t, \Mom_t, \Liv_t) \right\}_{t \in \Tim}
\mathstop
$$
For convenience we have $\Par$ denote the union of all $\Par_t$, or equivalent terms,
$$
\Par(h) := \Par_{\Fert(h)}(h)
$$
and likewise for $\Mom$.

Each $\Par_t$ is a random function from a subset of $\Hap_t$ to $\Dip_{<t}$
where
$$
  \Dip_{<t} := \bigcup_{s < t} \Dip_s
\mathstop
$$

Each $\Mom_t$ is a random function from the domain of $\Par_t$ to subsets of $\Loc$.
For a haploid index $h \in \dom \Par_t$,
$\Mom_t(h)$ represents the genome locations inherited from the maternal haploid genome of the
parent $\Par_t(h)$ producing the gamete indexed by $h$.
It follows that $(\Loc - \Mom_t(h))$ is the subset of locations passed down from
paternal haploid genomes of parent $\Par_t(h)$.


The values of $\Liv_t$ represent all the mature individuals living at time $t$.
They are random finite subsets of $\Dip_{\le t}$ where
$$
  \Dip_{\le t} := \bigcup_{s < t} \Dip_s
\mathstop
$$
The living times of every diploid $d \in \Dip$,
$$
\{ t : d \in \Liv_t \}
$$
must be an interval relative to $T$ (i.e. any time in $T$ that is between two members of
an interval must also be in that interval).


\section{Haploid lineages}

A haploid lineage process induces a random function $\Lin$ which maps
pairs of genomic location and descendant haploid to a lineage of all the 
haploids transmitting genetic information at that genomic location to the descendant haploid.
For every genomic location $\ell \in \Loc$ and haploid $h \in \Hap$,
$$
\Lin(\ell, h) := \bigcup_{i} \{ A_i(\ell, h) \}
$$
where $A$ is defined inductively as follows:
$$
A_0(\ell, h) := h
$$
and for non-negative integers $i$,
$$
A_{i+1}(\ell, h) := (\Par(A_i(\ell, h)), s)
$$
where $s = 0$ (maternal) if $\ell \in \Mom(A_i(\ell, h))$
otherwise $s = 1$ (paternal).



\section{An embedded Ancestral Recombination Graph}

An ancestral recombination graph \cite{friedman_ancestral_1997}
\cite{hein_gene_2005} \cite{wakeley_coalescent_2009} of a sampled
population is embedded in any outcome of any haploid lineage process.
We formally show the
exact embedding using the gARG formalism \cite{wong_what_arg_2022}.

We start by defining the \emph{genetic legacy} of a haploid
$h \in \Hap$ for sample population $S \subseteq \Hap$ to be
$$
  \Leg(h, S) := \{ (\ell, d) \in \Loc \times S : h \in \Lin(\ell, d) \}
\mathstop
$$

This genetic legacy is the genetic material that survives in the
sample population $S$ originally copied from ancestral haploid $h$
(with or without mutations).

Genetic legacy for a sample population $S$ induces the following
equivalence relationship over pairs of haploid $h_1$ and $h_2$ in
$\Hap$:
$$
h_1 \simeq_S h_2 \ := \ \Leg(h_1, S) = \Leg(h_2, S)
\mathstop
$$
We denote the resulting equivalence class containing $h \in \Hap$ as
$$
{[h]}_S \ := \ \{ h' : \Leg(h', S) = \Leg(h, S) \}
\mathstop
$$

In this equivalence relationship, gametes are considered equivalent if
they have the same genetic legacy for the sample population $S$.

A convenient choice for an embedded gARG \cite{wong_what_arg_2022} is
to set the gARG nodes (vertices) to be the equivalence classes:
$$
   \Nodes(S) := \{ {[h]}_S : h \in \Hap \}
\mathstop
$$

The (unannotated) graph edges of the gARG are chosen as child-parent
node pairs $(C, P) \in \Nodes(S)^2$ where

\begin{quote}
TODO
\end{quote}

$$
  (\Par(h), i) \in P \text{ for some $h \in C$ and some $i \in \{0, 1\}$ .}
$$

In the gARG, annotations are added for each graph edge (pair of child
and parent nodes). This annotation is the set of locations through which
genetic information has been copied from parent to child. In the
following interpretation, the only locations of interest are those for
which genetic information has been transmitted into the sample
population $S$. With this interpretation, the annotation for edge
$(C,P)$is
$$
  \{ \ell \in \Loc :
     C \cup P \subseteq \Lin(\ell, h) \text{ for some $h \in S$}
  \}
\mathstop
$$

\section{Acknowledgements}

Thanks to Daria Shipilina and Nick Barton for sharing their preprint
\cite{shipilina_origin_2022} and discussing the conjecture in edition
0.1 of this document relating to their preprint.

\section{Changes from edition 0.1}

\begin{itemize}
\item
  add section about embedded ARG
\item
  removed conjecture relating to \cite{shipilina_origin_2022}
\end{itemize}


\printbibliography
\end{document}
