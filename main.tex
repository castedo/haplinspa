\documentclass{article}
\usepackage{amsmath, amssymb}
\usepackage{hyperref}
\usepackage[backend=biber, style=numeric]{biblatex}
\addbibresource{references.bib}

\begin{document}


\newcommand{\mathstop}{\text{ .}}
\renewcommand{\Pr}{\operatorname{\mathbb{P}}}
\newcommand{\dom}{\operatorname{dom}}
\newcommand{\rng}{\operatorname{rng}}
\newcommand{\Hap}{\mathrm{Hap}}
\newcommand{\Dip}{\mathrm{Dip}}
\newcommand{\Gam}{\mathsf{Gam}}
\newcommand{\Mate}{\mathsf{Mate}}
\newcommand{\Par}{\mathrm{Par}}
\newcommand{\Fert}{\mathrm{Fert}}
\newcommand{\Loc}{\mathrm{Loc}}
\newcommand{\Lin}{\mathsf{Lin}}
\newcommand{\Leg}{\mathsf{Leg}}
\newcommand{\Nodes}{\mathsf{Nodes}}


\begin{abstract}
\textbf{STAGE:} WORKING DRAFT

\textbf{DOCUMENT TYPE:} Formal Mathematical Definition

\textbf{OBJECTIVES:} EQUATIONS

\begin{itemize}
\item
  Formal mathematical definition of stochastic process used by
  statistical estimator of admixture timing under development.
\item
  Precise mathematical definitions for technical discussions relating to
  ancestral recombination graphs.
\end{itemize}
\end{abstract}


\section{Haploid index space}

A haploid index space serves as a fixed non-stochastic basis over which additional
stochastic structures are built.
A specific haploid index is merely a placeholder for many possible gametes and haploid
genomes. It indirectly represents a specific gamete or haploid genome only given an
outcome of a stochastic process.

A \emph{Haploid index space} is a tuple
\[
  (T, \Hap, \Dip, \Fert, \mathcal{A}, \mu)
\]
with components
\begin{itemize}
\item
  $T$, an index set of numerical times (e.g. integers or reals),
\item
  $\Hap$, a set of (haploid) indices to distinct gametes,
\item
  $\Dip$, a set of pairs of haploid indices of egg and sperm gametes that fuse,
\item
  $\Fert$, a mapping from $\Dip$ to fertilization time,
\item
  $\mathcal{A}$, a $\sigma$-algebra over $\Hap$, and
\item
  $\mu$, a measure on $\mathcal{A}$ (which can be a simple count or a kind of ''mass'' of
haploids).
\end{itemize}

For convenience, given a \emph{haploid index space},

\begin{itemize}
\item
  $\Hap_0$ denotes the set of indices to egg gametes,
\item
  $\Hap_1$ denotes the set of indices to sperm gametes,
\item
  $\Dip_*$ denotes the mapping from haploid indices to the containing pair in $\Dip$,
and
\item
  $\Fert_*$ denotes the function composition $\Fert \circ \Dip_*$.
\end{itemize}

Formally, a \emph{haploid index space} must satisfy the following
conditions.

\begin{enumerate}
\item
  $\Hap_0 \cup \Hap_1 = \Hap$ and $\Hap_0 \cap \Hap_1 = \emptyset$.
\item
  $\Dip \subset \Hap_0 \times \Hap_1$ and forms a one-to-one mapping
  between $\Hap_0$ and $\Hap_1$,
\item
  $\Fert_*$ is a measurable function relative to $\mathcal{A}$ (and the Borel algebra)
\end{enumerate}


\section{Haploid lineage process}

A haploid index space is fixed and not random. But it can be extended, with a
probability space, to a stochastic process.

Given 
\begin{itemize}
\item
  a haploid index space $(T, \Hap, \Dip, \Fert, \mathcal{A}_H, \mu)$,
\item
  a measurable space $(\Loc, \mathcal{A}_L)$ of genomic locations,
\item
  and probability space $(\Omega, \mathcal{F}, \Pr)$,
\end{itemize}
a \emph{Haploid lineage process} is a stochastic process defined as a time-indexed family
of three random variables
$$
 \left\{ (S_t, \Par_t, M_t) \right\}_{t \in T}
\mathstop
$$

The values of $S_t$ are sets of haploid indices, specifically, measurable sets in
$\mathcal{A}$. These represent all the haploid genomes of mature individuals living at
time $t$. Various conditions much hold such as fertilization times being less than or
equal to $t$ and once an individual ceases to be living, it can not rise from the dead.

The values of $\Par_t$ are functions from a subset of $S_t$ to $\Dip$
with the following conditions:
\begin{enumerate}
\item
  every $\Par_t$ is measurable in terms of $\mathcal{A}$ and $\mathcal{F}$,
\item
  for each $t \in T$, the domain of $\Par_t$ ($\dom \Par$) is a subset of $S_t$, and
\item
  for every haploid $h \in \Hap$,
$$
  \Fert_*(h) > \Fert(\Par(h))
\mathstop
$$
\end{enumerate}
where $\Par$ is the union of all $\Par_t$.


\begin{quote}
TODO: pick a better name than $M_t$ for the following function
\end{quote}

Lastly, $M_t$ is a measurable function from $\Hap$ to subsets of $\Loc$
representing the genome locations inherited from the maternal haploid genome of the
parent generating a gamete.
It follows that $(\Loc - M_t)$ is the subset of locations passed down from
paternal haploid genomes of parents generating a gamete.


\section{Random indices}

Given a measure $\mu$ defined on $\mathcal{A}$,
random index variables $\{ S_t \}_{t \in T}$ taking values in $\dom \Par_t$ uniformly
(per $\mu$)
with probability one conditional on $\Par_t$.



\section{Gametic genealogy}

\begin{quote}
TODO: updated to be stochastic process on top of haploid index space.
\end{quote}

A \emph{gametic genealogy} is a convenient mathematical formalism of the
genealogy of a population from the perspective of gametes.
Mathematically, it is a quadruple \[
(\Gam, \Mate, \Par, \Fert) 
\] with components

\begin{itemize}
\item
  \(\Gam\), the set of underlying gametes,
\item
  \(\Mate\), the set of zygotes formed by the fusion of egg gametes and
  sperm gametes,
\item
  \(\Par\), a mapping from child gametes to parent zygotes, and
\item
  \(\Fert\), a mapping from zygotes to fertilization time.
\end{itemize}

For convenience, given a \emph{gametic genealogy},

\begin{itemize}
\item
  \(\Gam_0\) denotes the set of egg gametes,
\item
  \(\Gam_1\) denotes the set of sperm gametes, and
\item
  \(\Mate_*\) denotes the mapping from gametes to the zygotes they
  formed during fertilization.
\end{itemize}

Formally, a \emph{gametic genealogy} must satisfy the following
conditions.

\begin{enumerate}
\def\labelenumi{\arabic{enumi}.}
\item
  \(\Gam_0 \cup \Gam_1 = \Gam\) and \(\Gam_0 \cap \Gam_1 = \emptyset\).
\item
  \(\Mate \subset \Gam_0 \times \Gam_1\) and forms a one-to-one mapping
  between \(\Gam_0\) and \(\Gam_1\).
\item
  \(\Par\) is a function \(C \mapsto \Mate\), where \(C\) is a subset of
  \(\Gam\) representing child gametes.
\item
  \(\Fert\) is a function \(\Mate \mapsto \mathbb{R}\) such that for all
  child gametes \(g \in \dom \Par\), \[
  \Fert(\Mate_*(g)) > \Fert(\Par(g))
  \mathstop
  \] \(\dom \Par\) denotes the domain of \(\Par\), that is, the set of
  child gametes.
\end{enumerate}

\section{Gametic lineage space}

A \emph{gametic lineage space} is a mathematical formalism representing
the lines of transmission of genetic information via gametes of a
population over time. It is a triplet \[
(\Loc, G, \Lin) 
\] where

\begin{itemize}
\item
  \(\Loc\) is the set of all genomic locations,
\item
  \(G\) is a gametic genealogy \((\Gam, \Mate, \Par, \Fert)\), and
\item
  \(\Lin\) is a function \(\Loc \times \Gam \mapsto 2^\Gam\) mapping a
  genomic position in a gamete to the set of gametes that transmitted
  genetic information to that position in that gamete.
\end{itemize}

For every location \(\ell \in \Loc\) and gamete \(g \in \Gam\),
\(\Lin(\ell, g)\) is the lineage ending at gamete \(g\) via locus
\(\ell\) and it must satisfy the condition \[
\Lin(\ell, g) = \{g\} \cup \Lin(\ell, \Par(g)_i) \text{ for either $i=0$ or $i=1$}
\] when \(g \in \dom \Par\), otherwise \(\Lin(\ell, g) = \{g\}\).

\(\Par(g)_0\) and \(\Par(g)_1\) are the maternal and paternal gametes,
respectively, that fertilized the parent of \(g\).

\section{Stochastic gametic lineage space}

A \emph{stochastic gametic lineage space} is a gametic lineage space
extended to model a random gametic lineage.

From a stochastic gametic linage space, a time-indexed family of
probability distributions \(\{ P_t \}_{t \in I}\) is induced.

\begin{quote}
TO DO: Need to rework space formalism to clarify over what is the
\(\sigma\)-algebra:
\href{https://gitlab.com/castedo/study-docs/-/issues/42}{issues \#42}.
\end{quote}

For convenience we define the set of all lineages that contain a gamete
\(g \in \Gam\) as \[
B(g) := \{ \Lin(\ell, s) : g \in \Lin(\ell,s), \ell \in \Loc, s \in \Gam \}
\mathstop
\]

Formally, a \emph{stochastic gametic lineage space} is a quintuple \[
(G, I, \{ S_i \}_{i \in I}, \mathcal{F}, \mu)
\] where

\begin{itemize}
\item
  \(G\) is a gametic lineage space
  \((\Loc, (\Gam, \Mate, \Par, \Fert), \Lin)\)
\item
  \(I\) is an index set of points in time with \(\rng \Fert \subset I\),
\item
  \(\{S_t\}_{t \in I}\) is a time-indexed collection of sets of living
  zygotes,
\item
  \(\mathcal{F}\) is a \(\sigma\)-algebra (\(sigma\)-field) over
  \(\rng \Lin\), and
\item
  \(\mu\) is a measure on \(\mathcal{F}\)
\end{itemize}

which satisfy the following conditions

\begin{itemize}
\item
  \(B(g) \in \mathcal{F}\) for all \(g \in \Gam\), and
\item
  \(\mu(S_t)\) is defined and finite for all \(t \in I\).
\end{itemize}

Every gametic lineage space induces a time-indexed family
\(\{P_t\}_{t \in I}\) of probabilities spaces measurable on
\(\sigma\)-algebra \(F\). This defines the probability of lineages which
end in a zygote alive at time \(t\).

\begin{quote}
TO DO: Need to clarify relationship between \(F\) and \(\Loc\) and
\(\Mate\) for when they are uncountable.
\end{quote}

\section{An embedded Ancestral Recombination Graph}

An ancestral recombination graph \cite{friedman_ancestral_1997}
\cite{hein_gene_2005} \cite{wakeley_coalescent_2009} of a sampled
population is embedded in a gametic lineage space. We formally show the
exact embedding using the gARG formalism \cite{wong_what_arg_2022}.

We start by defining the \emph{genetic legacy} of a gamete
\(g \in \Gam\) for sample population \(S \subseteq \Gam\) to be \[
  \Leg(g, S) := \{ (\ell, d) \in \Loc \times S : g \in \Lin(\ell, d) \}
\mathstop
\] This genetic legacy is the genetic material that survives in the
sample population \(S\) originally copied from ancestral gamete \(g\)
(with or without mutations).

Genetic legacy for a sample population \(S\) induces the following
equivalence relationship over pairs of gametes \(g_1\) and \(g_2\) in
\(\Gam\): \[
g_1 \simeq_S g_2 \ := \ \Leg(g_1, S) = \Leg(g_2, S)
\mathstop
\] We denote the resulting equivalence class containing \(g \in \Gam\)
as \[
{[g]}_S \ := \ \{ g' : \Leg(g', S) = \Leg(g, S) \}
\mathstop
\]

In this equivalence relationship, gametes are considered equivalent if
they have the same genetic legacy for the sample population \(S\).

A convenient choice for an embedded gARG \cite{wong_what_arg_2022} is
to set the gARG nodes (vertices) to be the equivalence classes: \[
   \Nodes(S) := \{ {[g]}_S : g \in \Gam \}
\mathstop
\]

The (unannotated) graph edges of the gARG are chosen as child-parent
node pairs \((C, P) \in \Nodes(S)^2\) where \[
  \Par(g)_i \in P \text{ for some $g \in C$ and some $i \in \{0, 1\}$ .}
\]

In the gARG, annotations are added for each graph edge (pair of child
and parent nodes). This annotation is the set of locations through which
genetic information has been copied from parent to child. In the
following interpretation, the only locations of interest are those for
which genetic information has been transmitted into the sample
population \(S\). With this interpretation, the annotation for edge
\((C,P)\) is \[
  \{ \ell \in \Loc :
     C \cup P \subseteq \Lin(\ell, g) \text{ for some $g \in S$}
  \}
\mathstop
\]

\section{Acknowledgements}

Thanks to Daria Shipilina and Nick Barton for sharing their preprint
\cite{shipilina_origin_2022} and discussing the conjecture in edition
0.1 of this document relating to their preprint.

\section{Changes from edition 0.1}

\begin{itemize}
\item
  add section about embedded ARG
\item
  removed conjecture relating to \cite{shipilina_origin_2022}
\end{itemize}


\printbibliography
\end{document}
