\documentclass{article}
\usepackage{amsmath, amssymb}
\usepackage{hyperref}
\usepackage[backend=biber, style=numeric]{biblatex}
\addbibresource{references.bib}

\begin{document}


\newcommand{\mathstop}{\text{ .}}
\renewcommand{\Pr}{\operatorname{\mathbb{P}}}
\newcommand{\dom}{\operatorname{dom}}
\newcommand{\Tim}{\mathrm{Tim}}
\newcommand{\Hap}{\mathrm{Hap}}
\newcommand{\Dip}{\mathrm{Dip}}
\newcommand{\Par}{\mathrm{Par}}
\newcommand{\Pat}{\mathrm{Pat}}
\newcommand{\Fert}{\mathrm{Fert}}
\newcommand{\Loc}{\mathrm{Loc}}
\newcommand{\Lin}{\mathsf{Lin}}
\newcommand{\Leg}{\mathsf{Leg}}
\newcommand{\Nodes}{\mathsf{Nodes}}


\begin{abstract}
\textbf{STAGE:} WORKING DRAFT

\textbf{DOCUMENT TYPE:} Formal Mathematical Definition

\textbf{OBJECTIVES:}

\begin{itemize}
\item
  Formal mathematical definition of stochastic process used by
  statistical estimator of admixture timing under development.
\item
  Precise mathematical definitions for technical discussions relating to
  ancestral recombination graphs.
\end{itemize}
\end{abstract}


\section{Notation}

\begin{itemize}
\item
  "$\dom f$" denotes the domain of function $f$
\item
  "$⟦ x \in S ⟧$" is Iverson bracket notation for the indicator function of $S$, namely
$$
⟦ x \in S ⟧ =
  \begin{cases}
    1 & \text{if } x \in S  \\
    0 & \text{otherwise.}
  \end{cases}
$$

\end{itemize}


\section{Fertilization Function}

A \emph{haploid lineage process} is formally defined in terms of a given
fixed \emph{fertilization function}, which we denote with the symbol $\Fert$. This
function maps possible occurrences of fertilization to points in time.

From a given fertilization function $\Fert$, we define three convenient symbols:
\begin{itemize}
\item
  $\Tim$: the range of $\Fert$ (a set of real numbers representing points in time),
\item
  $\Dip$: the domain of $\Fert$ (a set of possible diploid organisms), and
\item
  $\Hap$: the set $\Dip \times \{0, 1 \}$ representing fertilizing gametes.
\end{itemize}

For a diploid $d \in \Dip$, the members $(d, 0)$ and $(d, 1)$ of $\Hap$ index the
fertilizing egg and sperm gametes, respectively.

For convenience, we map $\Hap$ to fertilization times with:
$$
  \Fert_H((d, s)) := \Fert(d)
$$
for all $d \in \Dip$ and $s \in \{0, 1\}$.

We denote the following inverse images as:
$$
\begin{aligned}
\Dip_t & := \{ d \in \Dip : \Fert(d) = t \}  \\
\Hap_t & := \{ d \in \Hap : \Fert_H(d) = t \}  \\
\Dip_{<t} & := \{ d \in \Dip : \Fert(d) < t \}
\mathstop
\end{aligned}
$$

A technical requirement on any given $\Fert$ in this document is that $\Dip_t$ must be
countable for every $t \in \Tim$.

An example of a valid fertilization function is $\Fert : \mathbf{N}^2 \mapsto \mathbf{N}$
where $\Fert( (t, n) ) = t$.


\section{Haploid lineage process}

Given 
\begin{itemize}
\item
  a fertilization function $\Fert$,
\item
  a set $\Loc$ of genomic locations,
\item
  and probability space $(\Omega, \mathcal{F}, \Pr)$,
\end{itemize}
a \emph{Haploid lineage process} is a stochastic process defined by a time-indexed
family of two random variables
$$
 \left\{ (\Par_t, \Pat_t) \right\}_{t \in \Tim}
\mathstop
$$

Each $\Par_t$ is a random function from a subset of $\Hap_t$ to $\Dip_{<t}$.
$\Par_t(h)$ represents the diploid \textbf{Par}ent that produced the haploid gamete $h$.

Each $\Pat_t$ is a random function from the domain of $\Par_t$ to subsets of $\Loc$.
$\Pat_t(h)$ represents the genome locations replicated into the gamete $h$
from the \textbf{Pat}ernal haploid genome of the parent $\Par_t(h)$
(inherited from the father of the parent producing the gamete).

For convenience we define:
$$
\begin{aligned}
\Par(h) & := \Par_t(h)  \\
\Pat(h) & := \Pat_t(h)
\end{aligned}
$$
where $t = \Fert_H(h)$.


\section{Haploid lineages}

A haploid lineage process induces a random function $\Lin$ which maps a genomic location
in a descendant haploid genome to a haploid lineage.
A haploid lineage consists of all the haploids transmitting genetic information via a
genomic location to a descendant haploid.
For every genomic location $\ell \in \Loc$ and haploid $h \in \Hap$, its haploid
lineage is
$$
\Lin(\ell, h) := \bigcup_{i} \{ a_i \}
$$
where $a_i$ is defined inductively as follows:
$$
a_0 := h
$$
and for integers $i > 0$,
$$
a_{i+1} :=
  \begin{cases}
    \left( \Par(a_i), ⟦ \ell \in \Pat(a_i) ⟧ \right) & \text{if } a_i \in \dom \Par  \\
    a_i & \text{otherwise.}
  \end{cases}
$$


\section{An embedded Ancestral Recombination Graph}

An ancestral recombination graph \cite{friedman_ancestral_1997}
\cite{hein_gene_2005} \cite{wakeley_coalescent_2009} of a sampled
population is embedded in any outcome of any haploid lineage process.
We formally show the
exact embedding using the gARG formalism \cite{wong_what_arg_2022}.

We start by defining the \emph{genetic legacy} of an ancestral haploid
$h \in \Hap$ for sample population $S \subseteq \Hap$ to be
$$
  \Leg(h, S) := \{ (\ell, d) \in \Loc \times S : h \in \Lin(\ell, d) \}
\mathstop
$$

This genetic legacy is the genetic material that survives in the
sample population $S$ originally copied from ancestral haploid $h$
(with or without mutations).

Genetic legacy for a sample population $S$ induces the following
equivalence relationship over pairs of haploids $h_1$ and $h_2$ in
$\Hap$:
$$
h_1 \simeq_S h_2 \ := \ \Leg(h_1, S) = \Leg(h_2, S)
\mathstop
$$
We denote the resulting equivalence class containing $h \in \Hap$ as
$$
{[h]}_S \ := \ \{ h' : \Leg(h', S) = \Leg(h, S) \}
\mathstop
$$

In this equivalence relationship, haploids are considered equivalent if
they have the same genetic legacy for the sample population $S$.

A convenient choice for an embedded gARG \cite{wong_what_arg_2022} is
to set the gARG nodes (vertices) to be the equivalence classes:
$$
   \Nodes(S) := \{ {[h]}_S : h \in \Hap \}
\mathstop
$$

The (unannotated) graph edges of the gARG are chosen as child-parent
node pairs $(C, P) \in \Nodes(S)^2$ where
$$
  (\Par(h), i) \in P \text{ for some $h \in C$ and some $i \in \{0, 1\}$ .}
$$

In the gARG, annotations are added for each graph edge (pair of child
and parent nodes). This annotation is the set of locations through which
genetic information has been copied from parent to child. In the
following interpretation, the only locations of interest are those for
which genetic information has been transmitted into the sample
population $S$. With this interpretation, the annotation for edge
$(C,P)$ is
$$
  \{ \ell \in \Loc :
     C \subseteq \Lin(\ell, h) \text{ and } P \subseteq \Lin(\ell, h) \text{ for some $h \in S$}
  \}
\mathstop
$$

\section{Acknowledgements}

Thanks to Daria Shipilina and Nick Barton for sharing their preprint
\cite{shipilina_origin_2022} and discussing the conjecture in edition
0.1 of this document relating to their preprint.

\section{Changes from edition 0.1}

\begin{itemize}
\item
  add section about embedded ARG
\item
  removed conjecture relating to \cite{shipilina_origin_2022}
\end{itemize}


\printbibliography
\end{document}
